%! Author = Mark
%! Date = 07.12.2025

% Preamble
\documentclass[sigconf]{acmart}
\usepackage{csquotes}
\usepackage{graphicx}
%%
%% \BibTeX command to typeset BibTeX logo in the docs
\AtBeginDocument{%
  \providecommand\BibTeX{{%
    Bib\TeX}}}
%
%
\title{An Introduction to the Border Gateway Protocol and how Malicious Actors Exploit
its Weaknesses}
%
\keywords{BGP, Border Gateway Protocol, Origin Hijacking, Path Hijacking, RPKI, BGPSec}
%
\author{Mark Klement}
\affiliation{%
  \institution{Goethe Universität}
  \city{Frankfurt}
  \country{Germany}}
\email{s6840520@stud.uni-frankfurt.de}
%
%
%
%
\begin{document}
%
\begin{abstract}
%
Das Border Gateway Protokoll (kurz BGP) ist eines der wichtigsten Protokolle für das
moderne Internet.
%
Es dient der Erstellung von routing Tabellen, die verwendet werden, um Pakete und Daten
auch in beliebig großen Netzwerken möglichst effizient zwischen Absendern und Empfängern
zu vermitteln.
%
Ziel dieses Papers ist es die systematischen Schwächen des Protokolls
zu beleuchten.
%
Es soll die Frage geklärt werden, welche Ursachen den Schwächen des Protokolls
zugrunde liegen und wie Malicious Actors sich diese zu Nutzen machen.
%
Zu diesem Zweck werden wir zunächst auf die Herkunft und Anforderungen an das Protokoll,
dessen Entwicklung und den Stellenwert für moderne, verteilte Informationssysteme
eingehen.
%
Nach der Klärung der Grundlagen erfolgt eine Beschreibung der zwei
Hauptangriffsvektoren \enquote{Origin Hijacks} und \enquote{Path Hijacks}.
%
Wir werden anhand eines realen, möglichst anschaulichen Vorfalls verdeutlichen,
wie die beschriebenen Angriffsvektoren für das Protokoll in seiner Reinform ohne
zusätzliche Sicherheitsmechanismen zum Problem werden können.
%
Darauf aufbauend werden wir anschließend sowohl die gängigen Sicherheitsmaßnahmen, die
der Prävention dienen, als auch moderne Hilfsmittel für die Erkennung und Analyse von
aktiven und vergangenen Incidents erklären.
%
Abschließend werden wir die aktuelle Gesamtsituation bewerten und versuchen einen
kleinen Ausblick auf zukünftige Entwicklungen zu geben.
%
\end{abstract}
%
\maketitle
%
%
%
%
\section{Introduction}\label{sec:Introduction}
%
Gateway Protokolle lassen sich allgemein in zwei Klassen einteilen, die sich im
Wesentlichen darin unterscheiden, ob sie sich für das Routing innerhalb eines
Autonomen Systems (kurz AS) oder zwischen verschiedenen Autonomen Systemen eignen.
%
Innerhalb eines Autonomen Systems werden sogenannte Interior Gateway
Protokolle (kurz IGPs) verwendet, während für das Routing zwischen Autonomen Systemen,
auch genannt \enquote{Inter Domain Routing}, Exterior Gateway Protokolle (kurz EGPs)
verwendet werden.
%
\textit{Vorsicht: EGP kann sowohl eine Klasse von Protokollen als auch ein bestimmtes
Protokoll bezeichnen.
Das Protokoll EGP ist quasi der Vorgänger des BGP.}
%
Das BGP existiert sowohl in der Version iBGP für Intra Domain Routing als auch als
eBGP für Inter Domain Routing.
%
Innerhalb dieses Papers werden wir uns mit der Version für das Inter Domain Routing
beschäftigen und diese allgemein als BGP bezeichnen.
%
\par
\noindent
%
%Mit der fortschreitenden Verbreitung des Internets und der damit einhergehenden
%Zunahme an Größe und Komplexität stieg die Notwendigkeit einen Nachfolger für das
%ursprüngliche EGP zu finden.
%
Die sich heute im Einsatz befindliche Variante des BGP ist eine erweiterte Version des
BGP-4~\cite{rekhter2006border} dessen aktuelle Spezifikation zu diesem Zeitpunkt
fast 20 Jahre alt ist.
%
Das Hauptziel bei der Entwicklung des BGP war es, ein Protokoll zu erschaffen, das
zuverlässig in besonders großen und komplexen Netzwerken funktioniert.
%
Die wichtigsten Anforderungen waren entsprechend eine hohe Performance bei quasi
unbegrenzter Skalierbarkeit sowie eine hohe Zuverlässigkeit und Flexibilität bei der
Nutzung des Netzwerkes.
%
Das BGP sollte sowohl die Durchsetzung von Routing Policies der ISPs ermöglichen, als
auch in der Lage sein, Veränderungen oder Ausfälle von Routen dynamisch und dezentral
zu kompensieren.
%
Tatsächlich erfüllt das BGP diese Anforderungen so gut, dass es heute praktisch das
einzige für Inter Domain Routing genutzte Protokoll ist.
%
\par
\noindent
%
Unglücklicherweise resultieren aus diesen Anforderungen aber auch die größten Schwächen
des Protokolls~\cite{murphy2006bgp}.
%
Diverse Sicherheitsmaßnahmen wie kryptografische Verschlüsselungen, Ausstellung von
Zertifikaten oder ähnliches kosten wertvolle Systemressourcen~\cite{li2018bgp}.
%
Je nach Verfahren kann es sich bei diesen Ressourcen um Speicherplatz, Rechenleistung
oder auch Bandbreite handeln.
%
Was in kleinen Netzwerken möglicherweise noch vertretbare Aufwände sind, kann in einem
sehr großen Netzwerk oder gar dem globalen Internet sehr schnell zu einem
unüberwindbaren Hindernis werden.
%
Sicherheit, Effizienz und Geschwindigkeit sind in der Praxis oft Anforderungen, die nur
schwer gleichzeitig erfüllbar sind.
%
Entsprechend sind in der \enquote{Basisversion} des BGP keine nennenswerten
Sicherheitsmechanismen integriert, da diese den anderen Anforderungen an das
Protokoll im Wege stehen würden.
%
Lediglich die Nutzung von TCP als Transport-Layer-Protokoll bietet minimalen Schutz
vor Wiretapping Angriffen bei bereits etablierten Verbindungen zwischen zwei
Routern~\cite{rekhter2006border}.
%
Es hindert Angreifer aber nichts daran eine neue Verbindung zu einem Border-Router
aufzubauen und diesem dann gänzlich erfundene BGP-Nachrichten zu senden.
%
\par
\noindent
%
Die Einfachheit das auf Vertrauen basierende Prinzip auszunutzen gepaart mit der
Tatsache, dass die BGP-Router wie Weichen für große Mengen an Traffic funktionieren,
machen dieses Protokoll extrem interessant für Hacker.
%
Durch das Verbreiten falscher BGP Informationen können auf einen Schlag riesige Mengen
an Traffic umgeleitet werden.
%
Um BGP zu sichern sind daher zusätzliche Maßnahmen und Protokolle wie
SBGP~\cite{kent2002secure}, BGPSec~\cite{lepinski2017bgpsec} oder
RPKI~\cite{lepinski2012infrastructure} notwendig.
%
%
%
\section{Threat Assessment}
%
Dieses Kapitel soll ein etwas besseres Verständnis dafür vermitteln, welche Ziele
besonders lukrativ sind und ob es sich bei BGP-Hijacking gemessen an der Anzahl der
bekannten Vorfälle wirklich um ein ernstzunehmendes Problem handelt.
%
\newline
\newline
%
\textbf{Wer wird Ziel}
%
\newline
%
BGP-Hijacking hat generell eher große Unternehmen aus der Wirtschaft oder Regierungsnahe
Institutionen zum Ziel.
%
Privatpersonen sind in der Regel nur indirekt betroffen, wenn Beispielsweise die Daten
eines großen Unternehmens abgefangen werden und dazu auch Kundendaten gehören.
%
Die Kriterien, warum man als Unternehmen oder Organisation zur Zielscheibe wird sind
vielfältig.
%
Die Motive der Angreifer reichen von der Absicht der Sabotage von Services
oder Netzwerken~\cite{youtubehijack:2008} (DOS Angriffe) bis hin zu staatlich
organisierter Spionage.
%
Aber auch Angriffe aus finanziellen Gründen sind insbesondere seit dem Aufleben von
Kryptowährungen wie Bitcoin keine Seltenheit mehr~\cite{apostolaki2017hijacking}.
%
\newline
\newline
%
\textbf{Frequenz der Angriffe}
%
\newline
%
Es ist nicht einfach Eine genaue Anzahl tatsächlicher Angriffe zu benennen, da nicht
alle Angriffe als solche erkannt werden und es entsprechend eine gewisse Dunkelziffer
gibt.
%
Die Tendenz zeigt aber, dass mit dem generellen Wachstum des Internets und der Verlegung
von immer mehr Services und Diensten in das Internet auch die Zahl der Vorfälle
zunimmt~\cite{hijack_frequencies10078883}
%
Auch moderne Sicherheitsmaßnahmen wie RPKI und BGPSec scheinen nicht in der Lage zu
sein die absolute Anzahl der jährlichen Vorfälle zu senken.
%
Obwohl auch sie keinen garantierten Schutz vor erfolgreichen Angriffen bieten, können
sie dennoch helfen Angriffe gegen das eigene Netzwerk oder Unternehmen zumindest deutlich
zu erschweren, sofern sie korrekt implementiert werden.
%
\newline
\newline
%
\textbf{Einschätzung der Netzwerkbetreiber}
%
\newline
%
Eine Interessante Studie~\cite{sermpezis2018survey} aus 2018 mit 75 befragten
Netzwerkbetreibern hat ergeben, dass BGP-Hijacking von den Netzwerkbetreibern als ein
ernstzunehmendes Problem bezeichnet wird.
%
Mehr als 40\% der Befragten Betreiber gaben an bereits zum Ziel von Hijacking
Angriffen geworden zu sein.
%
Die Dauer der Angriffe betrug in 57\% der Fälle mehr als eine Stunde, in 25\% der
Fälle sogar mehr als einen Tag.
%
Etwa 71\% gaben damals an unter anderem aus Kostengründen kein RPKI zu verwenden und
setzten stattdessen eher auf verstärktes Peering mit anderen Netzwerken um die
Auswirkungen von Hijacking Incidents abzumildern.
%
%
%
\section{Austausch von BGP Informationen}
%
In seiner ursprünglichen Form funktioniert der Austausch von BGP Daten wie bereits
erwähnt auf Vertrauensbasis, da das Unterlassen einer stringenten Prüfung der
versendeten/ erhaltenen Daten beträchtliche Mengen an Systemleistung einspart.
%
Die dedizierten BGP Router, die die ASes miteinander verbinden informieren sich also
gegenseitig darüber, mit welchen ASes sie jeweils verbunden sind.
%
In einer Kette von 3 ASes (Bspw. AS 1, AS 2 und AS 3 mit AS 2 in der Mitte) würde AS 2
die jeweils Anderen darüber informieren, dass AS 1 und AS 3 via AS 2 miteinander
kommunizieren können.
%
\begin{figure}[!h]
%\centering
\includegraphics[width=0.5\textwidth]{images/basic_bgp_route_announcements}
\caption{ASes exchanging Routing information}
\label{fig:BGP Info Exchange}
\end{figure}
%
\par
\noindent
%
BGP-4 unterstützt CIDR (Classless Inter Domain Routing) sowohl für IPv4 als auch für
IPv6~\cite{rekhter2006border}.
%
Es gibt 4 Nachrichtentypen, wovon 3 der Verwaltung der Verbindung zu unmittelbaren
Nachbarn (Peers) dienen.
%
Der eigentliche Austausch der Routen erfolgt über Nachrichten vom Typ \enquote{UPDATE}.
%
Routen können sowohl bekannt gegeben, als auch zurückgezogen werden.
%
\par
\noindent
Die wichtigsten Attribute innerhalb der BGP-Nachrichten sind:
%
\begin{itemize}
    \item Die \enquote{AS\_Number}, die der eindeutigen Identifikation von Autonomen
    Systemen dient
    \item Ein \enquote{AS\_Path} bestehend aus einer Kette von AS\_Numbers, der zu
    einem anderen, erreichbaren AS führt
    \item Die \enquote{NRLI} Network Layer Reachability Information, also der Prefix des
    Adressbereichs, zu dem der AS\_Path führt
\end{itemize}
%
%\newline
%
Es gibt zwar noch weitere Attribute, aber im Wesentlichen lässt sich der Internetgraph
aus den oben genannten Attributen erstellen, weshalb wir es für den Kontext dieses
Papers dabei belassen.
%
%
%
\section{Kategorien von Angriffen}
%
Manche Quellen~\cite{jaw2024serial} teilen BGP Hijacking incidents grob in die Kategorien
\enquote{Prefix Hijacks} und \enquote{Path Hijacks} ein.
%
Andere Quellen nehmen diese Unterscheidung nicht vor und bezeichnen alle Arten von
Vorfällen als \enquote{BGP Prefix Hijacks}~\cite{sermpezis2018survey}.
%
Manche Quellen nehmen sogar noch feinere Einteilungen vor und zerlegen Beispielsweise
die Klasse der Prefix Hijacks in \enquote{Black-Hole-Attacks} und
\enquote{Interception-Attacks}~\cite{oesterle2021challenges}.
%
Teileweise wird auch gemessen daran, ob eine böse Absicht vorliegt, zwischen \enquote{BGP
Hijacks} und eher unabsichtlichen Fehlkonfigurationen, dann of genannt \enquote{BGP
Leaks}, unterschieden~\cite{hijack_frequencies10078883}.
%
\par
\noindent
%
Insbesondere Quellen die weniger strikten wissenschaftlichen Standards folgen und die im
Internet auf Webseiten wie Wikipedia oder in Blogs zu finden sind, vermischen die
Bezeichnungen wie \enquote{Prefix Hijacking}, \enquote{Origin Hijacking},
\enquote{BGP Hijacking}, \enquote{Route Hijacking} und \enquote{Path Hijacking} komplett.
%
\par
\noindent
%
Insgesamt haben sowohl unsere eigene Literaturrecherche als auch die anderer Autoren
ergeben, dass die Kategorisierungen und Bezeichnungen von Angriffen teilweise
inkonsistent sind~\cite{raynor2022state}.
%
In diesem Paper werden wir deshalb die verhältnismäßig gängige Einteilung in
\enquote{Prefix Hijacks} und \enquote{Path Hijacks} verwenden, je nachdem, ob der Angriff
sich hauptsächlich über den Adressraum selbst oder über den Pfad zu einem entsprechenden
Adressraum definiert.
%
\subsection{Prefix Hijacking}
%
Generell bevorzugen Router bei der Erstellung ihrer Routingtabellen Announcements mit
möglichst präzisen Prefixes~\cite{jaw2024serial}.
%
Ein Announcement wird immer dann präziser, wenn der Prefix in der CIDR Notation länger,
also die Zahl nach dem \enquote{\slash} größer wird.
%
\newline
\newline
%
\textbf{Example in IPv4:}
%
\newline
\newline
%
\texttt{192.232.0.0\slash22} \hspace{10pt} $\rightarrow$ \hspace{10pt} less precise
%
\newline
\newline
%
\texttt{192.232.0.0\slash24} \hspace{10pt} $\rightarrow$ \hspace{10pt} more precise
%
\newline
\newline
%
Angreifer Nutzen diese sogenannte \enquote{longest-prefix-matching-rule} für sich
aus, indem sie präzisere Prefixes in ihren BGP-Nachrichten verschicken als die
legitimen Eigentümer eines Adressraumes.
%
So überreden sie quasi andere Border-Router ihre gefälschten Announcements zu
akzeptieren und zu bevorzugen.
%
\begin{figure}[!h]
%\centering
\includegraphics[width=0.5\textwidth]{images/prefix_specific_hijack}
\caption{Prefix hijack}
\label{fig:BGP prefix hijack}
\end{figure}
%
\par
\noindent
%
Je präziser die Prefixes in den Announcements sind, desto kleiner ist automatisch der
Adressraum, der in dem Prefix enthalten ist.
%
Das kann für einen Angreifer einen Nachteil darstellen, wenn er einen möglichst großen
Adressraum übernehmen will und die Empfänger seiner Announcements alle zu
unspezifischen Announcements verwerfen.
%
Um sein Ziel zu erreichen muss er dann eine auffällig hohe Menge an sehr spezifischen
Announcements versenden, was schnell dazu führen kann, dass der Angriff als solcher
erkannt wird~\cite{Musawi_survey:2016}.
%
\par
\noindent
%
Wenn ein legitimer Besitzer eines Adressraumes den Verdacht hat, dass er das Opfer eines
Prefix-Hijacks ist, kann er diese Methode tatsächlich genauso anwenden wie der Angreifer
und dadurch versuchen seinen gekaperten Adressraum zurückzuerobern.
%
\par
\noindent
%
Dies war zum Beispiel 2008 der Fall, als Pakistan Telecom versucht hat die Verfügbarkeit
von YouTube innerhalb von Pakistan einzuschränken~\cite{ripe:2008}.
%
\par
\noindent
%
Dabei hat Youtube ursprünglich einen \slash22er Prefix verwendet, der dann von
Pakistan Telecom durch einen \slash24er Prefix gehijacked wurde.
%
YouTube versuchte zunächst den Adressraum zurückzuerobern, indem es ebenfalls einen
\slash24er Prefix announced hat.
%
Mit zwei Prefixen identischer Länge im Umlauf konnte YouTube jedoch nur einen Teil des
Traffics zurückerobern.
%
Schlussendlich musste YouTube zwei \slash25er Prefixes announcen, um den
vollständigen Adressraum des \slash24er Prefixes abzudecken und wieder den gesamten Traffic
zu erhalten.
%
\subsection{Path Hijacking}
%
Ein weiteres extrem relevantes Attribut bei der erstellung der Routingtabellen ist der
sogenannte \enquote{AS\_Path} und seine Länge.
%
Um Pakete möglichst schnell und zuverlässig zuzustellen werden Routen mit kurzem
AS\_Path gegenüber denen mit langem Pfad bevorzugt.
%
Die Grundidee ist, dass ein kürzerer Pfad weniger Hops bedeutet, was die Anzahl
der potenziellen points of failure auf der Route, den kommunikativen Overhead und auch
(wahrscheinlich) die Latenz reduziert~\cite{jaw2024serial}.
%
Angreifer Nutzen dieses Grundprinzip also aus, indem sie behaupten, dass sie einen
besonders kurzen Weg zu einem bestimmten Adressraum kennen.
%
\par
\noindent
%
\begin{figure}[!h]
%\centering
\includegraphics[width=0.5\textwidth]{images/path_hijack}
\caption{BGP Path Hijack}
\label{fig:BGP Path Hijack}
\end{figure}
%
Die Pfadlänge ist zwar wichtig, allerdings hat die \enquote{longest-prefix-matching-rule}
eine noch höhere Priorität.
%
Path Hijacks werden deshalb unter anderem dann interessant, wenn Angreifer aus
irgendeinem Grund (Beispielsweise wegen RPKI) nicht in der Lage sind, noch spezifischere
Announcements zu versenden als die legitimen Besitzer des Adressraumes.
%
Wenn die Angreifer nur in der Lage sind einen Prefix der gleichen Länge zu versenden
wie die legitimen Besitzer eines Adressraumes, dient die Pfadlänge als tie-breaker.
%
In diesem Fall wird der Teil des Traffics an den Angreifer umgeleitet, der von
Absendern kommt, die näher an dem AS des Angreifers liegen als an dem des
tatsächlichen Besitzers des Adressraumes.
%
Auch diese Situation konnte 2008 während dem Youtube Hijacking~\cite{ripe:2008}
beobachtet werden, als der gleiche \slash24er Prefix sowohl von Pakistan Telecom als auch
von Youtube announced wurde.
%
Hat ein Angreifer nicht einmal die Möglichkeit den gleichen Prefix wie ein legitimer
Besitzer zu announcen, kann er durch die Behauptung einen besonders kurzen Pfad zum
Ziel zu kennen aber immer noch Versuchen zu erreichen, dass andere ASes ihren Traffic
dem Angreifer schicken, weil sie denken, dass der Angreifer ihn besonders schnell
weiterleiten könne.
%
In diesem Fall würde das AS des Angreifers dann nicht mehr vorgeben das Ziel des Traffics
zu sein, sondern lediglich als besonders attraktives Transit-AS auftreten.
%
\par
\noindent
%
Das Hauptziel des Angreifers ist es allerdings immer irgendwie in den Besitz des
Traffics zu kommen.
%
Ob er sich als Transit-AS oder als Ziel-AS ausgibt ist aus seiner Sicht in der Regel
zweitrangig, da er in beiden Fällen die gleichen Möglichkeiten hat, was er mit dem
erhaltenen Traffic anstellen kann.
%
Er kann ihn in der Rolle des Ziel-AS verwerfen oder ihn als Transit-AS einfach nicht
mehr weiterleiten.
%
Genauso gut kann er den Traffic nachdem er ihn beispielsweise inspiziert oder
modifiziert hat auch in beiden Fällen an den legitimen Empfänger weiterleiten, um
nicht so schnell aufzufallen und noch mehr Informationen abzugreifen.
%
%
%
\section{Sicherheitsmaßnahmen}
%
Die Sicherheitsmaßnahmen lassen sich im Wesentlichen in zwei Kategorien unterteilen
\cite{shapira2022ap2vec}.
%
Die erste Kategorie beinhaltet die präventiven Maßnahmen, die verhindern sollen, dass es
überhaupt zu Hijacks und Route Leaks kommt.
%
Zu diesen Methoden gehören unter anderem RPKI~\cite{lepinski2012infrastructure},
SBGP~\cite{kent2002secure}, BGPSec~\cite{lepinski2017bgpsec} und RouteFiltering.
%
Die zweite Kategorie umfasst die Maßnahmen, die der Erkennung von Hijacks dienen,
sobald oder nachdem diese auftreten~\cite{Musawi_survey:2016, raynor2022state,
hoarau2021suitability}.
%
In dieser Kategorie befinden sich Verfahren, die sich mit dem (Live)
Monitoring und der Anomalieerkennung beschäftigen~\cite{Holterbach:295685,
shapira2022ap2vec, buehler:2023, sermpezis2018artemis}.
%
\newline
\newline
%
\textbf{RPKI (Resource Public Key Infrastructure)}
%
\newline
%
RPKI ist ein kryptografisches Verfahren, das der Authentifizierung von Routing
Informationen dient~\cite{lepinski2012infrastructure}.
%
Dabei werden Kombinationen von AS Nummern und Präfixen durch Certificate
Authorities (CAs) in einer Kette signiert und validiert, sodass eine möglichst
stabile Trust-Chain mit einem Trust-Anchor am Anfang entsteht.
%
Auf diese Weise wird verhindert, dass Autonome Systeme Prefixes announcen, die sie
nicht kontrollieren, was folglich insbesondere die Vermeidung von
\enquote{Origin Hijacks} bewirken soll.
%
Ein aus Prefix und AS\_Number bestehender Datensatz wird auch als Route Origin
Authorization Object (ROA Objekt) bezeichnet, das für alle Border-Router zugänglich in
einer verteilten Datenbank liegt.
%
ROA Objekte können zusätzlich ein optionales Attribut enthalten, mit dem die maximal
erlaubte Länge eines Prefix announcements für einen gegebenen Adressraum spezifiziert
wird.
%
Dadurch soll verhindert werden, dass Angreifer RPKI aushebeln, indem sie korrekte
Kombinationen aus AS\_Number und Adressraum verwenden und den eigentlichen Besitzer
einfach mit einem oder mehreren, spezifischeren Subprefix announcements ausstechen.
%
Bei der Verwendung dieses Attributs müssen Besitzer von Adressräumen deshalb aufpassen,
dass sie die maximale Prefix Länge nicht höher ansetzen als es bei den Prefixen, die
in den entsprechenden ROA Objekten enthalten sind, der Fall ist~\cite{bush2014origin}.
%
Wenn diese Regel nicht eingehalten wird und ein AS Beispielsweise einen Prefix der Länge
\slash16 in ein ROA Objekt schreibt und das max\_Length Attribut dann aber auf \slash24
setzt, nennt man das \enquote{Loose ROA}.
%
Loose ROAs sind besonders gefährlich, wenn kein Verfahren zur Verifikation von
AS\_Paths eingesetzt wird, da Angreifer dann die Möglichkeit haben, einen sogenannten
\enquote{Forged-Origin Subprefix Hijack} durchzuführen, bei dem der Hijack sogar
durch eine eigentliche Sicherheitsmaßnahme (RPKI) validiert werden kann.
%
Obwohl die Verwendung von RPKI in seiner noch nicht vollständig standardisierten
Anfangszeit in 2011 recht selten war, hat sich diese Situation mittlerweile deutlich
verbessert und auch die Anzahl der Miskonfigurationen hat stark
abgenommen~\cite{chung2019rpki, jaw2024serial}.
%
\newline
\newline
%
\textbf{BGPSec}
%
\newline
%
Ähnlich wie RPKI ist auch BGPSec~\cite{lepinski2017bgpsec} ein kryptografisches
Verfahren.
%
Im Unterschied zu RPKI werden hier aber nicht die kombinationen aus AS\_Number und
Prefix signiert, sondern die Pfade innerhalb der Announcements.
%
Eines der Probleme ist, dass diese Signatur in jedem Schritt auf dem Pfad vom Origin AS
bis zum Ziel AS erfolgen muss.
%
Dadurch ist nicht nur der Rechenaufwand im Vergleich zu ROV mittels RPKI relativ hoch,
sondern es reicht auch noch ein einziger Router ohne BGPSec auf dem Pfad aus, um
potenzielle Sicherheitsvorteile zu verlieren.
%
Und selbst wenn BGPSec auf dem vollständigen Pfad erfolgreich implementiert wird,
demonstrieren~\cite{li2018bgp}, dass es trotzdem noch Möglichkeiten gibt Traffic
abzufangen.
%
Wenn zwei kompromittierte ASes miteinander kooperieren, können sie mittels einer
Remote-BGP Session einen direkten Link und damit auch einen sehr kurzen Pfad simulieren.
%
Sofern das AS, das auf dem Pfad näher am Ziel liegt den erhaltenen Pfad signiert, können
die folgenden ASes nicht feststellen, dass kein echter Link vorhanden ist und die
simulierte Verbindung wird gegenüber einer scheinbar längeren aber echten Route
bevorzugt.
%
\newline
%
Weiterhin kann über eine hohe Frequenz an Announcements und Withdrawals einer Route
durch einen Angreifer die Instabilität dieser Route vorgetäuscht werden.
%
Das kann der Angreifer mit allen Routen machen, bis die scheinbar stabilste und
attraktivste Route aus Sicht des Opfers durch das AS des Angreifers führt.
%
\newline
%
BGPSec ist zwar in der lage zumindest einige Sicherheitsbedenken auszuräumen, aber es
keine Optimallösung die vollständige Sicherheit
garantiert~\cite{li2018bgp, oesterle2021challenges}.
%
\newline
\newline
%
\begin{figure}[!h]
%\centering
\includegraphics[width=0.5\textwidth]{images/sec_scal_spd_diagram_v3}
\caption{Pure BGP vs. BGP with additional security measures (Graph Scale more
arbitrary than exact)}
\label{fig:BGP vs BGP with sec measures}
\end{figure}
%
\newline
\newline
%
\textbf{Route Filtering}
%
\newline
%
Route Filtering kann sowohl für eingehende, als auch ausgehende Announcements
angewendet werden.
%
Hierzu verwalten die BGP Router Listen von Routen oder Präfixen, die sie auf
bestimmten Ports erwarten und entsprechend akzeptieren.
%
Diese Informationen können von der Internet Routing Registry (IRR) bezogen werden.
%
\newline
%
Logischerweise ist es im Interesse der ISPs die eingehenden Announcements zu filtern,
um sich selbst vor Angriffen zu schützen.
%
Seriöse ISPs filtern aber auch ihre ausgehenden Announcements, um beispielsweise
fehlerhafte Announcements durch (versehentliche) Miskonfigurationen zu vermeiden.
%
Im Idealfall filtern alle ISPs beziehungsweise deren BGP Router sowohl die ein- als
auch die ausgehenden Announcements.
%
\newline
\newline
%
\textbf{Monitoring und Anomalieerkennung}
%
\newline
%
Beim Monitoring geht es primär um die Beobachtung von Veränderungen im gewöhnlichen
Betriebsablauf.
%
\par
\noindent
%
Speziell zu diesem Zweck entwickelt gibt es eine Reihe von Monitoring-Tools, die die
Identifikation von Incidents deutlich erleichtern sollen.
%
Beispiel hierfür sind Tools die der Visualisierung von Routen dienen, wodurch
plötzlich auftretende Veränderungen leichter erkennbar sind~\cite{raynor2022state}.
%
\par
\noindent
%
Andere Tools untersuchen die Menge an BGP-Traffic und die Frequenz von BGP-Update
Nachrichten~\cite{Musawi_survey:2016}.
%
Häufigkeitsanalysen können auch mit der Untersuchung anderer Features der Nachrichten
wie der Länge von Pfaden oder dem Auftreten von verdächtigen ASNs kombiniert werden,
um die Erkennungsrate zu erhöhen~\cite{Musawi_survey:2016}.
%
Eine besondere Herausforderung ist aber immer die Unterscheidung zwischen
\enquote{natürlichen} Vorfällen, die Beispielsweise durch Hardwareausfälle bedingt
sind und solchen Vorfällen, die tatsächlich auf das handeln böswilliger Akteure
zurückzuführen sind.
%
\newline
%
Um bösartige Angriffe von \enquote{natürlichen} Vorfällen zu unterscheiden, kann eine
Analyse der von einer Umleitung betroffenen Prefixes helfen~\cite{buehler:2023}.
%
Die meisten ASes announcen mehrere Prefixes wodurch im Falle eines gewöhnlichen
technischen Defektes auch der Traffic für all diese Prefixes umgeleitet wird.
%
Gezielte Angriffe konzentrieren sich aber oft nur auf ein Subset der Prefixes eines
ASes, wodurch letztendlich auch nur das entsprechende Subset an Traffic umgeleitet wird,
während der Traffic der nicht betroffenen Prefixes weiterhin die übliche Route nimmt.
%
Ein solcher Angriff lässt sich dann möglicherweise daran erkennen, dass die Pakete, die
an einen gehijackten Teil des Adressraumes gesendet werden eine höhere Latenz
beziehungsweise Round Trip Time haben~\cite{buehler:2023}.
%
\newline
\newline
%
\textbf{KI-Basierte Anomalieerkennung}
%
\newline
%
Wie in vielen anderen technischen Bereichen gibt es auch bei der Anomalieerkennung
eine zunehmende Auswahl an KI-basierten Methoden.
%
Das Internet lässt sich sehr gut als Graph modellieren, bei dem die (BGP) Router die
Knoten und die Verbindungen zwischen ihnen die Kanten darstellen.
%
Entsprechend gut lassen sich Graphenbasierte Machine Learning Architekturen auf das
Problem anwenden~\cite{hoarau2021suitability}.
%
\newline
%
Ein alternativer Ansatz ist ein NLP-basiertes Verfahren, das in Anlehnung an Word2Vec
AP2Vec~\cite{shapira2022ap2vec} getauft wurde.
%
Es basiert auf der Annahme, dass ASes verschiedene Rollen wie tier-1 oder tier-2
Provider oder auch cloud Provider besitzen und sich die Verteilung der Rollen auf
einer Route im Falle eines Angriffs stärker verändert als dies bei einer regulären
Routenänderung normal ist.
%
Das Verfahren betrachtet die Routen als Sätze deren Wörter den Rollen der ASes
auf den jeweiligen Ruten entsprechen.
%
Aufgrund des Embeddings dieser Sätze kann dann darauf geschlossen werden, ob es sich
bei dem gegebenen Satz beziehungsweise der entsprechenden Route um einen Hijack
handelt oder nicht.
%
%
%
\section{Conclusion}
%
Das BGP ist das Protokoll, das unser modernes Internet zusammenhält und daran wird
sich auf absehbare Zeit auch nichts ändern.
%
Aufgrund seiner Bedeutsamkeit und der scheinbar unaufhörlich wachsenden Größe des
Internets wird das Protokoll entsprechend auch in Zukunft ein lukratives Ziel für
Angreifer bleiben.
%
\newline
%
Obwohl BGP allein ein äußerst anfälliges Protokoll ist, gibt es mittlerweile recht
effektive Präventionsmaßnahmen wie RPKI, BGPsec und RouteFiltering, die Angriffe zwar
nicht gänzlich verhindern, aber zumindest erschweren können.
%
\newline
%
Bedauerlicherweise scheitert es aber in der Praxis häufig an der Implementierung und
Anwendung dieser Maßnahmen.
%
Die Gründe dafür reichen von mangelnden Kenntnissen und Fähigkeiten der
Administratoren bis hin zu finanziellen Überlegungen der Netzwerkbetreiber.
%
Während mangelnde Fachkenntnisse durch gezielte Schulungen von Administratoren und
Betreibern behoben werden können, lässt sich der finanzielle Aspekt nur schwer
eliminieren.
%
\newline
%
Möglicherweise besteht die Option besonders kostenoptimiert arbeitenden Betreibern zu
begegnen, indem ein gesetzlicher Rahmen geschaffen wird, der fahrlässiges Handeln unter
so hohe Strafen stellt, dass die Implementierung von Sicherheitsmaßnahmen die günstigere
Option ist.
%
Solche Gesetze international zu erlassen und anschließend auch durchzusetzen scheint
aber ein schwieriges Unterfangen darzustellen, insbesondere da einige Staaten
beispielsweise zwecks Spionage oder Zensur vermutlich kein hohes Interesse an der
vollständigen Schließung sämtlicher Sicherheitslücken haben.
%
Uns sind zu dem Zeitpunkt der Verfassung dieses Papers keine Anstrengungen dieser Art
bekannt, weder durch einzelne Länder noch durch Vereinigungen von Staaten oder Ländern
wie Beispielsweise den USA oder die EU.
%
%
%\newpage
\section{To Do}
%
\begin{itemize}
    \item Vollständige Übersetzung des Textes in Englisch
    \item Referenzen häufiger in den Text einbinden
\end{itemize}
%
%
%
%\newpage
%\hspace{10px}
%\newpage
%
%
%
\bibliographystyle{ACM-Reference-Format}
\nocite{*}
\bibliography{bibliography}
%
%\appendix
%
\end{document}