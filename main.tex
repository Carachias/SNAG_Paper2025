%! Author = Mark Klement
%! Date = 05.11.2025


\documentclass[11pt,a4paper]{article}
\usepackage{authblk}
\usepackage[hyperref]{templates/acl2021}
% \usepackage{times} % superseded by:
\usepackage{mathptmx}
\usepackage{latexsym}
\usepackage{graphicx}
\renewcommand{\UrlFont}{\ttfamily\small}

% This is not strictly necessary, and may be commented out,
% but it will improve the layout of the manuscript,
% and will typically save some space.
\usepackage{microtype}

\aclfinalcopy % Uncomment this line for the final submission
%\def\aclpaperid{***} %  Enter the acl Paper ID here

%\setlength\titlebox{7cm} %% nur für Variante 1
% You can expand the titlebox if you need extra space
% to show all the authors. Please do not make the titlebox
% smaller than 5cm (the original size); we will check this
% in the camera-ready version and ask you to change it back.

\usepackage{xspace}
\usepackage[english]{babel}
\usepackage[autostyle]{csquotes}
\usepackage[autolanguage]{numprint}
\usepackage[english]{isodate}
\usepackage{enumitem}
\usepackage{caption}
\usepackage{natbib}
\setlist[itemize]{nosep}

\newcommand\BibTeX{B\textsc{ib}\TeX}
\newcommand{\Vox}{VoxML Annotation Environment\xspace} % VoxML Annotation Survey
\newcommand{\ttlabanno}{TTLab VoxML Annotator\xspace}

\title{BGP Hijacking}


%% das scheint am float placement auch nichts zu ändern:
%\setcounter{topnumber}{2}
%\setcounter{bottomnumber}{2}
%\setcounter{totalnumber}{4}
%\renewcommand{\topfraction}{0.85}
%\renewcommand{\bottomfraction}{0.85}
%\renewcommand{\textfraction}{0.15}
%\renewcommand{\floatpagefraction}{0.7}


%% Variante 1:
\author[*1]{Mark Klement}
\affil[*]{Chair for Cybersecurity, Goethe-University Frankfurt}
\affil[1]{\texttt{s6840520@stud.uni-frankfurt.de}}


\date{05.11.2025}



\begin{document}
\maketitle
\begin{abstract}
%
Ziel dieses Papers ist es die systematischen Schwächen des Border Gateway Protokolls
(BGP) zu beleuchten.
%
Es soll die Frage geklärt werden, welche Ursachen den Schwächen des Protokolls
zugrunde liegen und welche Ansätze es zur Lösung dieser gibt.
%
Zunächst soll die Herkunft und der Stellenwert des Protokolls für moderne, verteilte
Informationssysteme erklärt werden.
%
Anschließend wird anhand eines realen Vorfalls verdeutlicht, wie die beschriebenen
Schwächen des Protokolls zum Problem werden können.
%
Schlussendlich folgt eine Beschreibung von Methoden und Mechanismen, die dabei helfen
sollen, das Protokoll und dessen Verwendung sicherer zu gestalten.
%
\end{abstract}
%
%
%
\section{Introduction}\label{sec:Introduction}
%
Das Border Gateway Protokoll (kurz BGP) ist eines der wichtigsten Protokolle für das
moderne internet.
%
Es dient der Erstellung von routing Tabellen, die verwendet werden, um Pakete und Daten
auch in beliebig großen Netzwerken möglichst effizient zwischen Absendern und Empfängern
zu vermitteln.
%
Im Wesentlichen besitzen Gateway Protokolle zwei verschiedene Ausprägungen, die sich
darin unterscheiden, ob es sich um das Routing innerhalb eines Autonomen Systems
oder zwischen verschiedenen Autonomen Systemen handelt.
%
Innerhalb eines Autonomen Systems (kurz AS) werden sogenannte Internal Gateway
Protokolle (kurz IGPs) verwendet, während für den Aufbau der Routing Tabellen zwischen
den Autonomen Systemen Exterior Gateway Protokolle (kurz EGPs) verwendet werden.
%
\par
\noindent
%
\textit{Vorsicht: EGP kann sowohl eine Klasse von Protokollen als auch ein bestimmtes
Protokoll bezeichnen.
Das Protokoll EGP ist quasi der Vorgänger des BGP.}
%
%
%
\section{Das Border Gateway Protokoll} \label{sec:das_BGP}
%
Mit der fortschreitenden Verbreitung des Internets und der damit einhergehenden
Zunahme an Größe und Komplexität stieg die Notwendigkeit einen Nachfolger für das
ursprüngliche EGP zu finden.
%
Das Hauptziel bei der Entwicklung des BGP war es ein Protokoll zu erschaffen, das
zuverlässig in besonders großen und komplexen Netzwerken funktioniert.
%
Das bedeutet, dass das Protokoll in Bezug auf seinen (kommunikativen) Overhead
möglichst effizient funktionieren soll und dabei gleichzeitig eine hohe
Flexibilität bei der Erstellung der Routingtabellen und Nutzung des Netzwerkes ermöglicht.
%
\par
\noindent
%
Tatsächlich ist genau diese Effizienz- und Flexibilitätsanforderung die Hauptursache
der Schwächen des Protokolls.
%
Diverse Sicherheitsmaßnahmen wie kryptografische Verschlüsselungen, ausstellung von
Zertifikaten oder ähnliches kosten wertvolle Systemressourcen.
%
Je nach Verfahren kann es sich bei diesen Ressourcen um Speicherplatz oder
Rechenleistung handeln.
%
Was in kleinen Netzwerken möglicherweise noch vertretbare Aufwände sind, kann in einem
sehr großen Netzwerk oder gar dem globalen Internet sehr schnell zu einem
unüberwindbaren Hindernis werden.
%
Sicherheit und Effizienz/ Geschwindigkeit sind in der Praxis oft zwei Anforderungen, die
nur schwer gleichzeitig erfüllbar sind.
%
Entsprechend waren in der \enquote{Basisversion} des BGP so gut wie keine
nennenswerten Sicherheitsmechanismen integriert, da diese der
Geschwindigkeitsanforderung an das Protokoll im Wege stehen würden.
%
%
%
\section{Austausch von BGP Informationen}
%
In seiner ursprünglichen Form funktioniert der Austausch von BGP Daten quasi auf
Vertrauensbasis, da das Unterlassen einer stringenten Prüfung der versendeten/
erhaltenen Daten beträchtliche mengen an Systemleistung einspart.
%
Die dedizierten BGP Router, die die ASes miteinander verbinden, informieren sich also
gegenseitig darüber, mit welchen ASes sie jeweils verbunden sind.
%
In einer Kette von 3 ASes (Bspw. AS 1, AS 2 und AS 3 mit AS 2 in der Mitte) würde AS 2
die jeweils Anderen darüber informieren, dass AS 1 und AS 3 via AS 2 miteinander
kommunizieren können.
%
So lange es keine Bad-Actors in diesem System gibt funktioniert dies sowohl zuverlässig
als auch effizient.
%
\begin{figure}[!h]
%\centering
\includegraphics[width=0.5\textwidth]{images/basic_bgp_route_announcements}
\caption{ASes exchanging Routing information}
\label{fig:BGP Info Exchange}
\end{figure}
%
\par
\noindent
%
Problematisch wird ein Verfahren auf Vertrauensbasis aber logischerweise immer dann,
wenn sich Akteure mit bösen Absichten in das System einschleichen.
%
Die Einfachheit diese Vertrauensbasis auszunutzen gepaart mit der Tatsache, dass die
BGP-Router wie Weichen für große Mengen an Traffic funktionieren, machen dieses
Protokoll extrem interessant für Hacker.
%
Durch das Verbreiten falscher BGP Informationen können auf einen Schlag riesige Mengen
an Traffic umgeleitet werden.
%
Traffic kann entweder manipuliert und/oder über unnötig lange Schleifen ans Ziel
geführt, oder sogar gänzlich ins Nichts geroutet werden.
%
Die Motive dafür sind vielfältig und reichen von der Absicht der Sabotage von Services
oder Netzwerken (DOS Angriffe) bis hin zu staatlich organisierter Spionage.
%
%
%
\section{Typen und Ablauf eines Angriffs}
%
\subsection{Spezifischeren Adressraum announcen}
%
Generell bevorzugen Router bei der Erstellung ihrer Routingtabellen möglichst präzise
Announcements.
%
Ein Announcement wird immer dann präziser, wenn der Prefix in der CIDR Notation länger
, also die Zahl nach dem \enquote{\slash} größer wird.
%
\newpage
\noindent
%
\textbf{Beispiel:}
%
\newline
\newline
%
\texttt{192.232.0.0\slash22} \hspace{10pt} $\rightarrow$ \hspace{10pt} less precise
%
\newline
\newline
%
\texttt{192.232.0.0\slash24} \hspace{10pt} $\rightarrow$ \hspace{10pt} more precise
%
\newline
\newline
%
\begin{figure}[!h]
%\centering
\includegraphics[width=0.5\textwidth]{images/prefix_specific_hijack}
\caption{Prefix hijack}
\label{fig:BGP prefix hijack}
\end{figure}
%
\par
\noindent
%
Je präziser die Prefixes in den Announcements sind, desto kleiner ist automatisch der
Adressraum, der in dem Prefix enthalten ist.
%
Das kann für einen Angreifer einen Nachteil darstellen, wenn er einen möglichst großen
Adressraum angreifen will aber die Empfänger seiner Announcements alle zu
unspezifischen Announcements verwerfen.
%
Um sein Ziel dann zu erreichen muss er eine auffällig hohe Menge an sehr spezifischen
Announcements versenden, was schnell dazu führen kann, dass der Angriff als solcher
erkannt wird.
%
\par
\noindent
%
Wie wir in dem im folgenden Kapitel beschriebenen Fallbeispiel erkennen werden, kann
diese Methode aber auch von legitimen ASes verwendet werden, um den gekaperten
Adressraum zurückzuerobern.
%
\subsection{Kürzeren AS\_Path announcen}
%
Ein weiteres extrem relevantes Attribut bei der erstellung der Routingtabellen ist der
sogenannte \enquote{AS\_Path} und seine Länge.
%
Das BGP soll möglichst effizient, aber auch flexibel und dynamisch sein.
%
Aus diesem Grund werden Routen mit kurzem AS\_Path gegenüber denen mit langem Pfad
bevorzugt.
%
Die Grundidee ist, dass ein kürzerer Pfad weniger Hops bedeutet, was die Anzahl
der potenziellen points of failure auf der Route, den kommunikativen Overhead und auch
(wahrscheinlich) die Latenz reduziert.
%
\par
\noindent
%
Nicht nur Angreifer, sondern auch legitime ASes können dieses Attribut
\enquote{manipulieren}.
%
Legitime ASes können den Pfad beispielsweise künstlich verlängern, indem sie ihre eigene
AS\_Number mehrfach hintereinander in den Pfad schreiben, um anderen mitzuteilen, dass
zwar ein Pfad durch ihr Netzwerk existiert, dieser in der Praxis aber nicht genutzt
werden soll.
%
Dieses Vorgehen dient oftmals der Umsetzung von routing Policies und wird \enquote{
Path prepending} genannt.
%
\par
\noindent
%
Angreifer können dieses Prinzip ausnutzen, indem sie behaupten, dass sie einen Weg zu
einem bestimmten AS kennen, der viel kürzer ist als alle anderen bisher bekannten Wege.
%
Um der Effizienzanforderung nachzukommen liegt es dann im Interesse der
anderen ASes diesen kurzen Pfad zu übernehmen und ihren Traffic dem Angreifer zu
schicken, in der Hoffnung, dass dieser sämtliche Pakete auf der kürzesten Route
weiterleitet.
%
\begin{figure}[!h]
%\centering
\includegraphics[width=0.5\textwidth]{images/path_hijack}
\caption{BGP Path Hijack}
\label{fig:BGP Path Hijack}
\end{figure}
%
\par
\noindent
%
Sobald der Angreifer erstmal im Besitz des Traffics ist, hat er viele Möglichkeiten
großen Schaden anzurichten.
%
Er kann diesen beispielsweise verwerfen und dadurch die Verfügbarkeit eines
Netzwerkes oder Dienstes stören (real world incident: Pakistan Telecom taking
down YouTube).
%
Ist das Ziel großangelegte Spionage, hat der Angreifer aber auch die Möglichkeit die
Pakete zu inspizieren/ manipulieren und dann an den eigentlichen Empfänger
weiterzuleiten, um noch mehr Informationen zu erhalten und dabei unbemerkt zu bleiben.
%
\section{Real world Incidents}
%
In diesem Kapitel werden wir den genauen zeitlichen Ablauf eines echten Vorfalls
beschreiben.
%
\newline
\newline
%
Es handelt sich um einen Versuch der Zensur des Internets durch die pakistanische
Regierung.
%
Ziel war die Blockade von YouTube innerhalb von Pakistan, die allerdings
unbeabsichtigt eskalierte und dabei über die Landesgrenzen von Pakistan hinaus
Einfluss auf die Verfügbarkeit der Webseite hatte.
%
\paragraph{Pakistan Telecom taking down YouTube}\footnote{\url{https://www.ripe.net/about-us/news/youtube-hijacking-a-ripe-ncc-ris-case-study/}}
%
Event Timeline:
%
\begin{itemize}[leftmargin=*]
    \item Before, during and after Sunday, 24 February 2008:
    AS36561 (YouTube) announces 208.65.152.0/22
    \item Sunday, 24 February 2008, 18:47 (UTC):
    AS17557 (Pakistan Telecom) starts announcing the more specific address space of
    208.65.153.0/24.
    Routers around the world receive the announcement, and YouTube traffic is
    redirected to Pakistan.
    \item Sunday, 24 February 2008, 20:07 (UTC):
    AS36561 (YouTube) starts announcing 208.65.153.0/24.
    With two identical prefixes in the routing system, BGP policy
    rules, such as preferring the shortest AS path, determine which route is chosen.
    This means that AS17557 (Pakistan Telecom) continues to attract some of YouTube's
    traffic.
    \item Sunday, 24 February 2008, 20:18 (UTC):
    AS36561 (YouTube) starts announcing 208.65.153.128/25 and 208.65.153.0/25.
    Because of the longest prefix match rule, every router that receives these
    announcements will send the traffic to YouTube.
    \par
    \noindent
    Here we can see how Youtube uses/ needs two more specific announcements with
    Prefixes of Length 25 to cover the full address space with a prefix of length 24.
    \item Sunday, 24 February 2008, 20:51 (UTC):
    All prefix announcements, including the hijacked /24 which was originated by
    AS17557 (Pakistan Telecom) via AS3491 (PCCW Global), are seen prepended by another
    17557.
    The longer AS path means that more routers prefer the announcement originated by
    YouTube.
    \item Sunday, 24 February 2008, 21:01 (UTC):
    AS3491 (PCCW Global) withdraws all prefixes originated by AS17557 (Pakistan
    Telecom), thus stopping the hijack of 208.65.153.0/24.
\end{itemize}
%
\section{Sicherheitsmaßnahmen}
%
Die Sicherheitsmaßnahmen lassen sich im Wesentlichen in zwei Kategorien unterteilen
\citep{shapira2022ap2vec}.
%
Zum einen gibt es die präventiven Maßnahmen, die verhindern sollen, dass es überhaupt
zu Hijacks und Route Leaks kommt.
%
Zu diesen Methoden gehören RPKI, BGPSec und RouteFiltering.
%
Die zweite Kategorie beinhaltet die Maßnahmen, die der Erkennung von Hijacks dienen,
sobald diese auftreten.
%
In dieser Kategorie befinden sich Verfahren, die sich mit dem Monitoring und der
Anomalieerkennung beschäftigen.
%
\newline
\newline
%
\textbf{RPKI (Resource Public Key Infrastructure)}
%
\newline
%
RPKI ist ein kryptografisches Verfahren, das der Authentifizierung von Routing
Informationen dient.
%
Dabei werden Kombinationen von AS Nummern und Präfixen signiert und validiert.
%
Auf diese Weise wird verhindert, dass Autonome Systeme Prefixes announcen, die sie
nicht kontrollieren.
%
Der entstandene Datensatz wird auch als Route Origin Authorization (ROA) bezeichnet.
%
Die Verwendung dieses VErfahrens liegt in der Verantwortung der Internet Service
Provider (ISPs).
%
\newline
\newline
%
\textbf{BGPsec}
%
\newline
%
Ähnlich wie RPKI ist auch BGPsec ein kryptografisches Verfahren.
%
Im Unterschied zu RPKI werden hier aber nicht die kombinationen aus AS\_Number und
Prefix signiert, sondern die Pfade innerhalb der Announcements.
%
Dass BGPsec aktiv ist, erkennt man daran, dass das Attribut \enquote{AS\_Path} in den
Announcements durch das Attribut \enquote{BGPsec\_Path} ersetzt wurde.
%
\newline
\newline
%
\textbf{Route Filtering}
%
\newline
%
Route Filtering kann sowohl für eingehende, als auch ausgehende Announcements
angewendet werden.
%
Hierzu verwalten die BGP Router Listen von Routen oder Präfixen, die sie auf
bestimmten Ports erwarten und entsprechend akzeptieren.
%
Diese Informationen können von der Internet Routing Registry (IRR) bezogen werden.
%
\newline
%
Logischerweise ist es im Interesse der ISPs die eingehenden Announcements zu filtern,
um sich selbst vor Angriffen zu schützen.
%
Seriöse ISPs filtern aber auch ihre ausgehenden Announcements, um beispielsweise
fehlerhafte Announcements durch (versehentliche) Miskonfigurationen zu vermeiden.
%
Im Idealfall filtern alle ISPs beziehungsweise deren BGP Router sowohl die ein- als
auch die ausgehenden Announcements.
%
\newline
\newline
%
\textbf{Monitoring und Anomalieerkennung}
%
\newline
%
Beim Monitoring geht es primär um die Beobachtung von Veränderungen im gewöhnlichen
Betriebsablauf.
%
\par
\noindent
%
Ein Beispiel hierfür wäre die plötzliche Veränderung einer Route.
%
Wird eine Veränderung beobachtet, gilt es den Grund dafür herauszufinden.
%
Möglicherweise handelt es sich nur um den Ausfall der gewöhnlichen Route aus Gründen
wie einem Hardwaredefekt an einem Router oder Stromausfall.
%
Da sich in einem solchen Fall die Route dynamisch an die gegebenheiten anpassen soll,
um Ausfallsicherheit zu garantieren, gäbe es hier vermutlich nichts zu befürchten und
das BGP erfüllt wie gewollt seinen Job.
%
Wenn aber keinerlei Probleme mit der ursprünglichen Route bekannt sind, kann das schon
ein Anlass sein die Veränderung der Route genauer zu untersuchen.
%
\newline
%
Ein Weiteres Beispiel einer Anomalie, die es zu untersuchen gilt, kann eine
ungewöhnliche Menge an Traffic, insbesondere in Form von BGP-Announcements sein.
%
Werden plötzlich ungewohnt hohe Mengen an Traffic durch das eigene Netzwerk geleitet,
kann dies ein Hinweis darauf sein, dass sich eine Route geändert hat und es gilt wie
zuvor beschrieben herauszufinden, warum dies der Fall ist.
%
Erhält der eigene Router plötzlich eine auffällig hohe Menge an BGP-Announcements,
ist dies möglicherweise ein Indiz dafür, dass ein anderes AS versucht das Routing
gezielt zu manipulieren.
%
Das passiert zum Beispiel dann, wenn ein AS einen großen Adressraum mit vielen, sehr
spezifischen Announcements mit langen Präfixen übernehmen möchte.
%
\newline
%
Um bösartige Angriffe von \enquote{normalen} Vorfällen zu unterscheiden, kann eine
Analyse der von einer Umleitung betroffenen prefixes
helfen\citep{buehler:2023}.
%
Die meisten ASes announcen mehrere Prefixes wodurch im Falle eines gewöhnlichen
technischen Defektes auch der Traffic für all diese Prefixe umgeleitet wird.
%
Gezielte Angriffe konzentrieren sich aber oft nur auf ein Subset der Prefixes eines
ASes, wodurch letztendlich auch nur das entsprechende Subset an Traffic umgeleitet wird,
während der Traffic der nicht betroffenen Prefixes weiterhin die übliche Route nimmt.
%
Ein solcher Angriff lässt sich dann möglicherweise daran erkennen, dass die Pakete, die
an einen gehijackten Teil des Adressraumes gesendet werden eine höhere Latenz
beziehungsweise Round Trip Time haben.
%
\newline
\newline
%
\textbf{KI-Basierte Anomalieerkennung}
%
\newline
%
Wie in vielen anderen technischen Bereichen gewinnen auch bei der Anomalieerkennung
KI-basierte Methoden zunehmend an Bedeutung.
%
Das Internet entspricht einem Graphen, bei dem die (BGP) Router die Knoten und die
Verbindungen zwischen ihnen die Kanten darstellen.
%
Entsprechend gut lassen sich Graphenbasierte Machine Learning Architekturen auf das
Problem anwenden\citep{hoarau2021suitability}.
%
\newline
%
Ein alternativer Ansatz ist ein NLP-basiertes Verfahren, das in Anlehnung an Word2Vec
AP2Vec\citep{shapira2022ap2vec} getauft wurde.
%
Es basiert auf der Annahme, dass ASes verschiedene Rollen wie tier-1 oder tier-2
Provider oder auch cloud Provider besitzen und sich die Verteilung der Rollen auf
einer Route im Falle eines Angriffs stärker verändert als dies bei einer regulären
Routenänderung normal ist.
%
Das Verfahren betrachtet die Routen als Sätze deren Wörter den Rollen der ASes
auf den jeweiligen Ruten entsprechen.
%
Aufgrund des Embeddings dieser Sätze kann dann darauf geschlossen werden, ob es sich
bei dem gegebenen Satz beziehungsweise der entsprechenden Route um einen Hijack
handelt oder nicht.
%
%
%
\section{Conclusion}
%
Das BGP ist das Protokoll, das unser modernes Internet zusammenhält und daran wird
sich auf absehbare Zeit auch nichts ändern.
%
Aufgrund seiner Bedeutsamkeit und der scheinbar unaufhörlich wachsenden Größe des
Internets wird das Protokoll entsprechend auch in Zukunft ein lukratives Ziel für
Angreifer bleiben.
%
\newline
%
Obwohl BGP allein ein äußerst anfälliges Protokoll ist, gibt es mittlerweile sehr
effektive Präventionsmaßnahmen wie RPKI, BGPsec und RouteFiltering.
%
Bedauerlicherweise scheitert es aber in der Praxis häufig an der Implementierung und
Anwendung dieser Maßnahmen.
%
Die Gründe dafür reichen von mangelnden Kenntnissen und Fähigkeiten der
Administratoren bis hin zu finanziellen Überlegungen der Netzwerkbetreiber.
%
Während mangelnde Fachkenntnisse durch gezielte Schulungen von Administratoren und
Betreibern behoben werden können, lässt sich der finanzielle Aspekt nur schwer
eliminieren.
%
Die letzte und vielleicht einzige Möglichkeit besonders kostenoptimiert arbeitenden
Betreibern zu begegnen wäre das Schaffen eines gesetzlichen Rahmens, der fahrlässiges
Handeln unter so hohe Strafen stellt, dass die Implementierung von
Sicherheitsmaßnahmen die günstigere Option ist.
%
Solche Gesetze international zu erlassen und anschließend auch durchzusetzen ist aber
schier unmöglich, insbesondere da einige Staaten beispielsweise zwecks Spionage oder
Zensur auch gar kein Interesse an der vollständigen Schließung sämtlicher
Sicherheitslücken haben.
%
%
%
\section{To Do}
%
\begin{itemize}[leftmargin=*]
    \item Vollständige Übersetzung des Textes in Englisch.
    %
    Erfolgt bei der späteren überarbeitung des Textes, bei der auch die
    Verbesserungsvorschläge aus dem Review mit einfließen.
    \item Einfügen der Referenzen an den entsprechenden Textstellen.
    \item Aussagekräftigen Titel für das Paper finden.
    \item BGPsec Part noch weiter ausarbeiten.
    Meinung Reviewer?
    \item Event Timeline für den YouTube Hijack noch etwas umschreiben und
    vereinfachen, da diese bis jetzt fast eine 1:1 Kopie aus der Originalquelle ist.
    Die AS Nummern tragen vermutlich wenig zur Verständlichkeit des Ablaufs bei.
    Meinung Reviewer?
\end{itemize}
%
%
%
\bibliographystyle{templates/acl_natbib}
\nocite{*}
\bibliography{bibtest}


%\appendix
%
%
\end{document}
